%%%%%%%%%%%%%%%%%%%%%%%%%%%%%%%%%%%%%%%%%%%%%%%%%%%%%%%%%%%%%%%%%%%%%%%%%%%%%%%%%%%%%%%%%
% LICENSE NOTICE (CC BY 4.0)
%
% Author/Creator: Alexander Menzel
% Copyright: 2025 MiraTherm
%
% This work is licensed under the Creative Commons Attribution 4.0 International License.
% License Text (URI): https://creativecommons.org/licenses/by/4.0/
%%%%%%%%%%%%%%%%%%%%%%%%%%%%%%%%%%%%%%%%%%%%%%%%%%%%%%%%%%%%%%%%%%%%%%%%%%%%%%%%%%%%%%%%%

\chapter{Introduction}
\label{chap:Introduction}
In this chapter, the purpose of this document and the context of the project are described.

\section{Document Purpose}
This document provides a requirements specification for software of a \ac{mcu} based radiator thermostat, which will be developed as part of a master project at Fulda University of Applied Sciences. This software should implement basic consumer functions and could be used as a base for research, development and production of smart heating controllers or thermostats.

\section{Project Context}
The master project will be realized as part of a bigger interdisciplinary development named ``MiraTherm Radiator Thermostat'', which includes the following areas:

\begin{itemize}
	\item \textbf{Mechanics}: Development of the thermostat's power transmission mechanism for proper function with commonly used radiator valves, followed by the design of an enclosure.
	\item \textbf{Control algorithms}: Engineering of control algorithms to be used by the thermostat.
	\item \textbf{Electronics}: Development of the thermostat's \ac{pcb} and its integration with mechanical components.
	\item \textbf{Software}: The subject of this work, development of the thermostat's software and its integration with \acs{pcb} components.
\end{itemize}