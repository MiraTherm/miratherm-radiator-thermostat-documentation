%%%%%%%%%%%%%%%%%%%%%%%%%%%%%%%%%%%%%%%%%%%%%%%%%%%%%%%%%%%%%%%%%%%%%%%%%%%%%%%%%%%%%%%%%
% LICENSE NOTICE (CC BY 4.0)
%
% Author/Creator: Alexander Menzel
% Copyright: 2025 MiraTherm
%
% This work is licensed under the Creative Commons Attribution 4.0 International License.
% License Text (URI): https://creativecommons.org/licenses/by/4.0/
%%%%%%%%%%%%%%%%%%%%%%%%%%%%%%%%%%%%%%%%%%%%%%%%%%%%%%%%%%%%%%%%%%%%%%%%%%%%%%%%%%%%%%%%%

\documentclass[conference,nofonttune]{IEEEtran}
\IEEEoverridecommandlockouts
%
%--Packages-------------------------------------------------------------------%
%
\usepackage[T1]{fontenc}
\usepackage{mathptmx}
\let\Bbbk\relax
\usepackage{cite}
\usepackage{amsmath,amssymb,amsfonts}
\usepackage{algorithmic}
\usepackage{graphicx}
\usepackage{textcomp}
\usepackage{xcolor}
\usepackage{orcidlink}
\usepackage{enumitem}
%
%--Customization--------------------------------------------------------------%
%
\def\BibTeX{{\rm B\kern-.05em{\sc i\kern-.025em b}\kern-.08em
    T\kern-.1667em\lower.7ex\hbox{E}\kern-.125emX}}

% Adjust enumerate and itemize indentation and spacing
\setlist[enumerate]{leftmargin=2.5em, labelsep=0.3em} % , topsep=0pt, itemsep=0pt
\setlist[itemize]{leftmargin=2.5em, labelsep=0.3em, label=\raisebox{0.0ex}{\textbullet}} % , topsep=0pt, itemsep=0pt
%
%--Document--------------------------------------------------------------------%
%
\begin{document}
%
%--Title and Authors-----------------------------------------------------------%
%
\title{Decoupling UI Logic in Embedded Systems: Technical Design of R-MVP-based Thermostat Software\\
% Note: Sub-titles are not captured in Xplore and should not be used
%\thanks{Identify applicable funding agency here. If none, delete this.}
}
%
% Note: The class file is designed for, but not limited to, six authors.
%
% Add 1\textsuperscript{st} before the name of the first author, if there are 
% multiple authors.
%
\author{\IEEEauthorblockN{Alexander Menzel\orcidlink{0009-0007-3904-0439}}
\IEEEauthorblockA{\textit{Department of Electrical Engineering} \\
\textit{Fulda University of Applied Sciences}\\
Fulda, Germany \\
alexander.menzel@et.hs-fulda.de
}
% \and
% \IEEEauthorblockN{2\textsuperscript{nd} Given Name Surname}
% \IEEEauthorblockA{\textit{dept. name of organization (of Aff.)} \\
% \textit{name of organization (of Aff.)}\\
% City, Country \\
% email address or ORCID}
}
%
\maketitle
%
%--Abstract and Keywords--------------------------------------------------------%
%
\begin{abstract}
Text\dots
\end{abstract}
%
\begin{IEEEkeywords}
keyword1, keyword2, keyword3, keyword4, keyword5
\end{IEEEkeywords}
%
%--Sections--------------------------------------------------------------------%
%

\section{Introduction}
Heating private living spaces is one of the most significant sources of $CO_2$ emissions. In Germany, a substantial portion of annual greenhouse gas emissions originates from this sector \cite{StatistischesBundesamt.16.07.2025} \cite{Umweltbundesamt.15.03.2022}. While intelligent heating control and smart home systems offer an average energy saving potential of between 8 and 19\% \cite{Kersken.2018}, the market is currently dominated by proprietary solutions. Consequently, there is a lack of open-domain projects that can serve as a foundation for research and development of smart heating controllers.

This paper presents the software development for a radiator thermostat prototype, realized as part of the interdisciplinary ``MiraTherm Radiator Thermostat'' project at Fulda University of Applied Sciences. The overarching project aims to create a complete device, encompassing mechanics, electronics, and control algorithms.

The primary objective of this work is to establish a solid software foundation for the thermostat's microcontroller-based hardware. While the long-term vision includes control algorithms and wireless connectivity (e.g., Matter-over-Thread), the central contribution of this paper lies in the architectural design of the application and its User Interface (UI). Specifically, we propose the application of the Routed-Model-View-Presenter (R-MVP) design pattern in order to decouple UI logic from hardware drivers and core system logic. This approach is used to implement basic consumer functions considering constraints of an embedded system and of the C programming language.

\section{Background} % and Related Work
Text\dots
\subsection{Smart Radiator Thermostats}
Text\dots
\subsection{MVPVM Pattern}
Text\dots

\section{Requirements}
Text\dots

\section{Technical Design}
Text\dots
\subsection{Overall Architecture}
Text\dots
\subsection{R-MVPVM Pattern}
Text\dots

\section{Implementation}
Text\dots

\section{Verification and Results}
Text\dots
\subsection{Test Environment}
Text\dots
\subsection{Results}
Text\dots


\section{Conclusion}
Text\dots

%
%--Example Content-------------------------------------------------------------%
%
% Paragraphs example:

% \paragraph{Paragraph 1} 
% Text\dots
% \paragraph{Paragraph 2} 
% Test citation: \cite{IEEEwebsite}.

% Table example:

% \begin{table}[htbp]
% \caption{Table Type Styles}
% \begin{center}
% \begin{tabular}{|c|c|c|c|}
% \hline
% \textbf{Table}&\multicolumn{3}{|c|}{\textbf{Table Column Head}} \\
% \cline{2-4} 
% \textbf{Head} & \textbf{\textit{Table column subhead}}& \textbf{\textit{Subhead}}& \textbf{\textit{Subhead}} \\
% \hline
% copy& More table copy$^{\mathrm{a}}$& &  \\
% \hline
% \multicolumn{4}{l}{$^{\mathrm{a}}$Sample of a Table footnote.}
% \end{tabular}
% \label{tab1}
% \end{center}
% \end{table}

% Figure example:

% \begin{figure}[htbp]
% \centerline{\includegraphics{figures/fig1.png}}
% \caption{Example of a figure caption.}
% \label{fig}
% \end{figure}

%
%--Appendices-------------------------------------------------------------------%
%
%\appendices

% Appendix A
%\section{Title of Appendix A}
%Text\dots

% Appendix B
%\section{}
%Text\dots

%--Acknowledgment---------------------------------------------------------------%
%\section*{Acknowledgment}
%Text\dots

%--Bibliography-----------------------------------------------------------------%
\bibliographystyle{IEEEtran}
\bibliography{./bibtex/bib/bibliography}

\end{document}
