%%%%%%%%%%%%%%%%%%%%%%%%%%%%%%%%%%%%%%%%%%%%%%%%%%%%%%%%%%%%%%%%%%%%%%%%%%%%%%%%%%%%%%%%%
% LICENSE NOTICE (CC BY 4.0)
%
% Author/Creator: Alexander Menzel
% Copyright: 2025 MitaTherm
%
% This work is licensed under the Creative Commons Attribution 4.0 International License.
% License Text (URI): https://creativecommons.org/licenses/by/4.0/
%%%%%%%%%%%%%%%%%%%%%%%%%%%%%%%%%%%%%%%%%%%%%%%%%%%%%%%%%%%%%%%%%%%%%%%%%%%%%%%%%%%%%%%%%

\chapter{Objectives and concept presentation}
\label{chap:Objectives and concept presentation}
%
The focus of the presented master project is the development of a modern software for a \ac{mcu}-based radiator thermostat and its contribution to the public domain. The master project will be realized as a part of a bigger interdisciplinary development, which includes:

\begin{itemize}
	\item \textbf{Mechanics}: Development of the thermostats power transmission for proper function with commonly used radiator valves, followed by the design of an enclosure. This work will be presumably realized by Anton Surikov and advised by Prof. Dr. Tobias Müller.
	\item \textbf{Control algorithms}: Engineering of control algorithms to be used by the thermostat.
	\item \textbf{Electronics}: Development of the thermostats \ac{pcb} and its integration with mechanical components, presumably realized by Thomas Schneider and advised by Prof. Dr. Daniel Schönherr.
	\item \textbf{Software}: The subject of this work, development of the thermostats software and its integration with \acs{pcb} components. It will be presumably realized by Alexander Menzel and advised by Prof. Dr. Uwe Werner.
\end{itemize}

In the first part of this project, a basic software for the device should be implemented including hardware drivers, main state machines and tasks. At the end of this part a scientific paper will be written as an \ac{ieee} report, describing the developed software architecture and design decisions.

The software should be designed ready for prospective integration of the control algorithms and \ac{ble}. The latter feature is supposed to be used for smart home applications, specifically for the integration of the thermostat in Home Assistant and/or for provisioning of a \ac{http} \ac{api} using a gateway.

The following points are consequently objectives of this master project:

\begin{itemize}[noitemsep]
	\item Development or integration of drivers for all radiator thermostat components.
	\item Design of the software according to basic consumer functions and hardware features.
	\item Implementation and tests of the designed software.
	\item Writing of an \acs{ieee} report describing program design.
\end{itemize}

