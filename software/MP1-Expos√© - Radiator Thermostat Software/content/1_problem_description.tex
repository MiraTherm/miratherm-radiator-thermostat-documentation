%%%%%%%%%%%%%%%%%%%%%%%%%%%%%%%%%%%%%%%%%%%%%%%%%%%%%%%%%%%%%%%%%%%%%%%%%%%%%%%%%%%%%%%%%
% LICENSE NOTICE (CC BY 4.0)
%
% Author/Creator: Alexander Menzel
% Copyright: 2025 MitaTherm
%
% This work is licensed under the Creative Commons Attribution 4.0 International License.
% License Text (URI): https://creativecommons.org/licenses/by/4.0/
%%%%%%%%%%%%%%%%%%%%%%%%%%%%%%%%%%%%%%%%%%%%%%%%%%%%%%%%%%%%%%%%%%%%%%%%%%%%%%%%%%%%%%%%%

\chapter{Problem description}
\label{chap:Introduction}
%
Heating is one of the most $CO_2$-intensive areas of human life. In Germany, around 210 million tons of $CO2$ of the general total of 762 million tons of $CO_2$ equivalents emitted in 2021 came from heating private living spaces. \cite{StatistischesBundesamt.16.07.2025} \cite{Umweltbundesamt.13.10.2025}

Effective heating control has an average saving potential of between 8 and 19\%, which can be achieved through the use of intelligent heating controllers and smart home systems. \cite{Kersken.2018} 

Whereas such control systems are sophisticated and widespread in developed countries such as Germany, most of them are fully proprietary. There is currently no project of public domain, which could be used as a base for research, development and production of smart heating controllers or thermostats.

The general aim of this master's project is to develop a modern software for a \ac{mcu}-based radiator thermostat and to contribute it to the public domain.