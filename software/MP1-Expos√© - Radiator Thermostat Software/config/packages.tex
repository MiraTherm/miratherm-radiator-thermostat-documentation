% Adjustment of the page layout
\usepackage[
automark, 					% automatically create chapter information in the header
headsepline, 				% separating line under the header
ilines 						% align separating line to the left
]{scrlayer-scrpage}
%
% Adaptation to the national language
\usepackage[english]{babel}
%
% Encoding
\usepackage{fontspec}
%
% Euro sign etc.
\usepackage{textcomp}
%
% Itemize and enuerate settings
\usepackage{enumitem}
%
% Subfigures
\usepackage{subfigure}
%
% Wrapfigures
\usepackage{wrapfig}
%
% Math
\usepackage[intlimits]{amsmath}
\usepackage{amssymb}
\usepackage{empheq}
\usepackage{nicefrac}
\usepackage{upgreek}
\usepackage{siunitx}
%
% Index
\usepackage{makeidx}
%
% Simple definition of line spacing and page margins etc.
\usepackage{setspace}
\usepackage{geometry}
%
% List of abbreviations
\usepackage[printonlyused]{acronym}
%
% for text wrapping around images
\usepackage{floatrow}
%
% simpler color definition
\usepackage[usenames,dvipsnames]{xcolor}
%
% Program code
\usepackage{listings}
%
% Custom colors
\usepackage[dvipsnames]{xcolor}
%
% Link URL
\usepackage{url}
%
% continuous numbering of footnotes
\usepackage{chngcntr}
%
% Tables
\usepackage{tabularx}
\usepackage{multirow}
\usepackage{longtable}
\usepackage{colortbl}
\usepackage{array}
\usepackage{ragged2e}
\usepackage{lscape}
%
% For line breaks in tables
\usepackage{makecell}
%
% Individual design of cells in tables
\usepackage{hhline}
%
%
% needed for the definition of custom commands
\usepackage{ifthen}
%
% defines i.a. the commands \todo and \listoftodos
\usepackage{todonotes}
%
% roman numerals
\usepackage{romannum}
%
% for using toprule
\usepackage{booktabs}
%
% Frames
\usepackage{framed, color}
\usepackage[most]{tcolorbox}
%
% Bibliography
\usepackage{csquotes}
\usepackage[style=ieee,sorting=none,maxnames=25]{biblatex}
% Breaking of (long) URLs at any character
\usepackage{xurl}
%
% Tikz
\usepackage{color}
\usepackage{transparent}
\usepackage{pgfplots}
%
% Caption
\usepackage{caption}
%
% Space after comma
\usepackage{ziffer}
%
% Table colors
\usepackage{colortbl}
\usepackage{hhline}
%
% package pgf plots
\usepackage{pgfplots}
%
% For the university logo on the title page
\usepackage{wallpaper}
\usepackage{watermark}
%
% To change the indentation in enumerations
\usepackage{enumitem}
%
% To extend the character set
\usepackage{newunicodechar}
%
% To add PDF as pages
\usepackage{pdfpages}
%
% PDF options
\usepackage[
bookmarks,
bookmarksopen=true,
%these color definitions highlight links in the PDF in color (comment out for printing)
colorlinks=true,
linkcolor=blue, 			% simple internal links
anchorcolor=black,			% anchor text
citecolor=blue, 			% references to bibliography in the text
filecolor=magenta, 			% links that open local files
menucolor=red, 				% Acrobat menu items
urlcolor=cyan,
%
plainpages=false, 			% for the correct creation of bookmarks
pdfpagelabels, 				% for the correct creation of bookmarks
hypertexnames=false, 		% for the correct creation of bookmarks
linktocpage 				% link page numbers instead of text in the table of contents
]{hyperref}